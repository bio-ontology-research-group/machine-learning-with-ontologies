\documentclass{beamer}
\usepackage{booktabs}
\usepackage{pdfpages}
\usepackage{mathtools}
\usepackage{enumerate}
\usepackage{multirow,tabularx}
\usepackage{booktabs}
\usepackage{pdfpages}
\usepackage{proof}
\usepackage{cancel}
\usepackage{chronology}
\usepackage{graphicx}
\usepackage{ulem}
\usepackage{amsmath}
\usepackage{amssymb}
\usepackage{color}
\usepackage{animate}
\usepackage{xr}

\PassOptionsToPackage{usenames,dvipsnames,svgnames}{xcolor}  
\usepackage{tikz}
\usepackage{tkz-graph}


\usepackage{wasysym}
\usepackage{proof}
\usepackage{cancel}
\usepackage{chronology}
\usepackage{graphicx}
\usepackage{ulem}
\usepackage{amsmath}
\usepackage{amssymb}
\usepackage{color}
\usepackage{xcolor}
\usepackage{soul}
%\usepackage{pstricks}
\setbeamertemplate{navigation symbols}{}

\newcommand{\norm}[1]{\left\lVert#1\right\rVert}
\newcommand{\el}{$\mathcal{EL}^{++}$}
\renewcommand{\Re}{\mathbb{R}}
\newcommand{\BigO}[1]{\ensuremath{\operatorname{O}\bigl(#1\bigr)}}
\newcommand{\myul}[2][blue]{\sethlcolor{#1}\hl{#2}\setulcolor{black}}

\newcommand<>{\cunderline}[3]{\only<#1>{#3}\only<#2>{\underline{#3}}}
\newcommand<>{\cem}[3]{\only<#1>{#3}\only<#2>{\ul{#3}}}
\newcommand<>{\cgray}[3]{\only<#1>{#3}\only<#2>{\textcolor{gray}{#3}}}
\newcommand<>{\colorize}[4]{\only<#1>{#4}\only<#2>{\textcolor{#3}{#4}}}

\setbeamertemplate{navigation symbols}{\insertslidenavigationsymbol}
%\setbeamertemplate{navigation symbols}{}
% \addtobeamertemplate{navigation symbols}{}{%
%     \usebeamerfont{footline}%
%     \usebeamercolor[fg]{footline}%
%     \hspace{1em}%
%     \insertframenumber/\inserttotalframenumber
% }

\mode<presentation>
{
\usecolortheme{crane}
%\useoutertheme{split}

\expandafter\def\expandafter\insertshorttitle\expandafter{%
  \insertshorttitle\hfill%
  \insertframenumber\,/\,\inserttotalframenumber}

\usefonttheme[onlysmall]{structurebold}
}
\renewcommand{\em}{\itshape}
\usepackage{pifont}
\definecolor{purple}{rgb}{1,0,1}
\definecolor{dred}{rgb}{0.7,0,0}
\definecolor{myred}{rgb}{1,0,0}
\definecolor{dblue}{rgb}{0,0,0.7}
\definecolor{dgreen}{rgb}{0,0.5,0}
\definecolor{myyellow}{rgb}{1,1,0}
\newcommand{\parents}[1]{parents(#1)}
\setbeamertemplate{itemize item}[ball]


% \mode<presentation>
% {
% \useinnertheme[shadow=true]{rounded}
% \useoutertheme{infolines}
% \usecolortheme{dove}
% \setbeamerfont{block title}{size={}}
% }

\title[Bio-Ontologies]{Semantic similarity and machine learning with ontologies}

\author{Robert Hoehndorf and Maxat Kulmanov}


\date{}

\begin{document}

\begin{frame}
  \titlepage
\end{frame}

\begin{frame}
  \frametitle{Before the tutorial}
  See \url{https://github.com/bio-ontology-research-group/ontology-tutorial}:
  \begin{itemize}
  \item install Docker (e.g.: {\tt apt-get install docker})
  \item {\tt docker pull coolmaksat/embeddings:latest}
%  \item {\tt docker pull leechuck/ontology-ml:latest}
  \item {\tt docker run -i -t -p 8888:8888 coolmaksat/embeddings /bin/bash -c "jupyter notebook --notebook-dir=/home/borg/ontology-tutorial/ --ip='0.0.0.0' --port=8888 --no-browser --allow-root"}
  \end{itemize}
\end{frame}

% \begin{frame}
% \frametitle{Overview}
% \tableofcontents
% \end{frame}

\section{Ontologies and graphs}

\begin{frame}
  \frametitle{Ontologies, machine learning, and AI}
  \begin{itemize}
  \item ontologies are ubiquitous
  \item rich formal characterization (axioms)
  \item how can they be used for (predictive) data analysis?
    \begin{itemize}
    \item ``fuzzy'', similarity-based search
    \item predictive analysis and machine learning
    \item background knowledge
    \end{itemize}
  \end{itemize}
\end{frame}

\begin{frame}
  \frametitle{Learning goals}
  \begin{itemize}
  \item machine learning with ontologies as {\em features} (or
    background knowledge)
  \item unsupervised or supervised:
    \begin{itemize}
    \item here: mostly unsupervised {\em feature} learning
    \item ``deep'' learning
    \end{itemize}
  \item focus on existing tools and methods
    \begin{itemize}
    \item Jupyter Notebooks and code examples
    \end{itemize}
  \item not covered:
    \begin{itemize}
    \item learning ontologies (axioms, definitions) from data
    \item (most) natural language processing
    \item reasoning with ontologies
    \item learning on ``knowledge graphs''
    \item machine learning theory
    \end{itemize}
  \end{itemize}
\end{frame}

\begin{frame}
  \frametitle{Agenda}
  \begin{itemize}
  \item Introduction: ontologies and graphs
  \item Semantic similarity
  \item Machine learning:
    \begin{itemize}
    \item syntactic
    \item graph-based
    \item model-theoretic
    \end{itemize}
  \end{itemize}
\end{frame}

\begin{frame}
  \frametitle{Preliminaries: ontologies}
  \begin{itemize}
  \item Specific artifacts expressing the intended meaning of a
    vocabulary in terms of primitive categories and relations
    describing the nature and structure of a domain of discourse
    \begin{itemize}
    \item in order to account for the competent use of vocabulary in
      real situations (such as annotations in databases, etc.)
    \end{itemize}
  \item the intended meaning of {\em primitive} categories and
    relations is expressed through axioms (axiomatic method, Tarski)
  \end{itemize}
\end{frame}

\begin{frame}
  \frametitle{Preliminaries: axioms}
  \begin{itemize}
  \item {\em classes} represent kinds of things in the world
    \begin{itemize}
    \item {\em Arm}, {\em Apoptosis}, {\em Influenza}, {\em Homo
        sapiens}, {\em Drinking behavior}, {\em Membrane}
    \end{itemize}
  \item {\em instances} of classes are individuals satisfying the
    classes' intension
    \begin{itemize}
    \item my arm, the influenza I had last year, one ethanol molecule, etc.
    \end{itemize}
  \item {\em relations} between instances arise from interactions,
    configurations, etc., of individuals
    \begin{itemize}
    \item my arm is {\bf part of} me, the {\bf duration of} my
      influenza was 10 days
    \end{itemize}
  \item {\em axioms} specify the conditions that instances of a class
    must satisfy
    \begin{itemize}
    \item every instance of {\em Hand} is a {\bf part of} an instance
      of {\em Arm}
    \end{itemize}
  \end{itemize}
\end{frame}

\begin{frame}
  \frametitle{Description Logics: overview}
  \begin{itemize}
  \item TBox: axioms pertaining to the terminology of the domain (classes)
  \item ABox: axioms stating facts (assertions) about the world
  \item RBox: axioms holding for relations
  \item Reasoning: derive implicitly represented knowledge (e.g.,
    subsumption)
  \item NB: a ``knowledge graph'' is an ABox + RBox
  \end{itemize}
\end{frame}

\begin{frame}
  \frametitle{Manchester OWL Syntax}
  \begin{table}[ht]
    \centering
    \begin{tabular}{|l|l|l|}
      DL Syntax & Manchester Syntax & Example \\
      \hline
      $C \sqcap D$ & C and D & Human and Male \\
      $C \sqcup D$ & C or D & Male or Female \\
      $\neg C$ & not C & not Male \\
      $\exists R.C$ & R some C & hasChild some Human \\
      $\forall R.C$ & R only C & hasChild only Human \\
      $(\geq n R.C)$ & R min n C & hasChild min 1 Human \\
      $(\leq n R.C)$ & R max n C & hasChild max 1 Human \\
      $(= n R.C)$ & R exactly n C & hasChild exactly 1 Human \\
      $\{a\} \sqcup \{b\} \sqcup ...$ & \{a b ...\} & \{John Robert
                                                      Mary\} \\
      \hline
    \end{tabular}
  \end{table}
\end{frame}


% \begin{frame}
%   \frametitle{Description Logic ALC: syntax}
%   \begin{definition}
%     Let $N_C$ be a set of concept names and $N_R$ be a set of relation
%     names, $N_C \cap N_R = \emptyset$. $\mathcal{ALC}$ concept
%     descriptions are inductively defined as:
%     \begin{itemize}
%     \item If $A \in N_C$, then $A$ is an $\mathcal{ALC}$ concept
%       description
%     \item If $C, D$ are $\mathcal{ALC}$ concept description, and $r
%       \in N_R$, then the following are $\mathcal{ALC}$ concept descriptions:
%       \begin{itemize}
%       \item $C \sqcap D$
%       \item $C \sqcup D$
%       \item $\neg C$
%       \item $\forall r.C$
%       \item $\exists r.C$
%       \end{itemize}
%     \end{itemize}
%   \end{definition}
%   \begin{itemize}
%   \item Use $\bot$ as abbreviation of $A \sqcap \neg A$, $\top$ as
%     abbreviation of $A \sqcup \neg A$
%   \end{itemize}
% \end{frame}

% Examples of concept descriptions, dl1.pdf, p8

% \begin{frame}
%   \frametitle{Description Logic ALC: semantics}
%   \begin{definition}
%     An interpretation
%     $\mathcal{I} = (\Delta^\mathcal{I}, \cdot^\mathcal{I})$ consists
%     of a non-empty domain $\Delta^\mathcal{I}$ and an interpretation
%     function $\cdot^\mathcal{I}$:
%     \begin{itemize}
%     \item $A^\mathcal{I} \subseteq \Delta^\mathcal{I}$ for all $A \in
%       N_C$,
%     \item $r^\mathcal{I} \subseteq \Delta^\mathcal{I} \times
%       \Delta^\mathcal{I}$ for all $r \in N_R$
%     \end{itemize}
%     The interpretation function is extended to $\mathcal{ALC}$ concept
%     descriptions as follows:
%     \begin{itemize}
%     \item $(C \sqcap D)^\mathcal{I} := C^\mathcal{I} \cap D^\mathcal{I}$
%     \item $(C \sqcup D)^\mathcal{I} := C^\mathcal{I} \cup D^\mathcal{I}$
%     \item $(\neg C)^\mathcal{I} := \Delta^\mathcal{I} - C^\mathcal{I}$
%     \item $(\forall r.C)^\mathcal{I} := \{ d \in \Delta^\mathcal{I} |
%       \mbox{for all } e \in \Delta^\mathcal{I}: (d,e) \in
%       r^\mathcal{I} \mbox{ implies } e \in C^\mathcal{I}\}$
%     \item $(\exists r.C)^\mathcal{I} := \{ d \in \Delta^\mathcal{I} |
%       \mbox{there is } e \in \Delta^\mathcal{I}: (d,e) \in
%       r^\mathcal{I} \mbox{ and } e \in C^\mathcal{I}\}$
%     \end{itemize}
%   \end{definition}
% \end{frame}

% \begin{frame}
%   \frametitle{Description Logic: terminologies}
%   \begin{itemize}
%   \item A concept definition is of the form $A \equiv C$ where
%     \begin{itemize}
%     \item $A$ is a concept name
%     \item $C$ is a concept description
%     \end{itemize}
%   \item A TBox is a finite set of concept definitions such that it
%     \begin{itemize}
%     \item does not contain multiple definitions, % A equiv B, A equiv C
%     \item does not contain cyclic definitions
%       % A equiv B and C, B equiv A and C
%     \end{itemize}
%   \item A {\em defined concept} occurs on the left-hand side of a
%     definition
%   \item A {\em primitive concept} does not occur on the left-hand side
%     of a definition
%     % See: axiomatic-deductive method!
%   \item An interpretation $\mathcal{I}$ is a model of a TBox
%     $\mathcal{T}$ if it satisfies all its concept definitions:
%     $A^\mathcal{I} = C^\mathcal{I}$ for all $A \equiv C \in \mathcal{T}$
%   \end{itemize}
% \end{frame}

% \begin{frame}
%   \frametitle{Description Logic: assertions}
%   \begin{itemize}
%   \item An assertion is of the form $C(a)$ (concept assertion) or
%     $r(a,b)$ (role assertion), where $C$ is a concept description, $r$
%     is a role, $a,b$ are individual names from a set $N_I$ of such
%     names
%   \item An ABox is a finite set of assertions
%   \item An interpretation $\mathcal{I}$ is a model of an ABox
%     $\mathcal{A}$ if it satisfies all its assertions:
%     \begin{itemize}
%     \item $a^\mathcal{I} \in C^\mathcal{I}$ for all $C(a) \in
%       \mathcal{A}$
%     \item $(a^\mathcal{I},b^\mathcal{I}) \in r^\mathcal{I}$ for all
%       $r(a,b) \in \mathcal{A}$
%     \end{itemize}
%   \end{itemize}
% \end{frame}

% \begin{frame}
%   \frametitle{Description Logic: Reasoning}
%   \begin{itemize}
%   \item Subsumption: Is $C$ a subconcept of $D$?
%     \begin{itemize}
%     \item $C \sqsubseteq_\mathcal{T} D$ iff $C^\mathcal{I} \subseteq
%       D^\mathcal{I}$ for all models $\mathcal{I}$ of $\mathcal{T}$
%     \end{itemize}
%   \item Satisfiability: Is the concept $C$ non-contradictory?
%     \begin{itemize}
%     \item $C$ is satisfiable w.r.t. $\mathcal{T}$ iff $C^\mathcal{I}
%       \not= \emptyset$ for some model $\mathcal{I}$ of $\mathcal{T}$
%     \end{itemize}
%   \item Consistency: Is the ABox $\mathcal{A}$ non-contradictory?
%     \begin{itemize}
%     \item $\mathcal{A}$ is consistent w.r.t. $\mathcal{T}$ iff it has
%       a model that is also a model of $\mathcal{T}$
%     \end{itemize}
%   \item Instantiation: Is $e$ an instance of $C$?
%     \begin{itemize}
%     \item $\mathcal{A} \models_\mathcal{T} C(e)$ iff $e^\mathcal{I}
%       \in C^\mathcal{I}$ for all models $\mathcal{I}$ of $\mathcal{T}$
%       and $\mathcal{A}$.
%     \end{itemize}
%   \end{itemize}
% \end{frame}

% \begin{frame}
%   \frametitle{Offtopic: knowledge graphs}
%   My favorite definition of ``knowledge graph'':\\
%   A knowledge graph is an ABox + RBox.
%   \begin{itemize}
%   \item ontologies are (mostly) the TBox!
%   \end{itemize}
% \end{frame}

\begin{frame}
  \frametitle{Ontologies provide background knowledge}
  \centerline{\includegraphics[width=.8\textwidth]{t-cell-aggregation.png}}
\end{frame}

\begin{frame}
  \frametitle{Ontologies provide background knowledge}
  \centerline{\includegraphics[width=.8\textwidth]{t-cell-activation.png}}
\end{frame}

% \begin{frame}
%   \frametitle{Ontologies provide background knowledge}
%   \centerline{\includegraphics[width=.95\textwidth]{moesin-1.png}}
%   \centerline{\includegraphics[width=.85\textwidth]{moesin-2.png}}
  
% \end{frame}

\begin{frame}
  \frametitle{Using background knowledge}
  \begin{block}{Problem statement (first attempt):}
    Given a set of entities (instances) within an ontology (DL
    theory). Can we discover/predict {\em new} relations between the
    entities, or between entities and classes in the ontology?
  \end{block}
  \pause
  \begin{itemize}
  \item what relations, and when is a fact ``new''?
  \pause
  \item what features are relevant?
    \begin{itemize}
    \item depends on the relation!
    \end{itemize}
  \pause
  \item finding new facts is only one (minor?) use case
    \begin{itemize}
    \item other uses: encode background knowledge for machine learning
      models; add new classes; expand definition; constrained
      learning; etc.
    \item computing ``similarity''
    \end{itemize}
  \end{itemize}
\end{frame}

\begin{frame}
  \frametitle{Semantic similarity: some examples}
  \begin{itemize}
  \item Are cyclin dependent kinases {\em functionally} more similar
    to lipid kinases or to riboflavin kinases? How about {\em
      phenotypically}?
  \item Which protein in the {\em mouse} is functionally most similar
    to the zebrafish {\em gustducin} protein?
  \item Which mouse knockout resembles {\em Bardet-Biedl Syndrome 8}?
  \item Are there mouse knockouts that resemble the side effects of
    diclofenac?
  \item Which genetic disease produces similar symptoms to ebola?
  \item Does functional similarity correlate with phenotypic
    similarity?
  \end{itemize}
\end{frame}

\begin{frame}
  \frametitle{Semantic similarity}
  semantic similarity measures:
  \begin{itemize}
  \item for words, terms, classes
  \item role of background knowledge:
    \begin{itemize}
    \item statistical/distributional semantics, large corpora
    \item ontologies: (graph) topology
    \end{itemize}
  \item similarity measures: hand-crafted or data-driven?
  \end{itemize}
\end{frame}

\begin{frame}
  \frametitle{Semantic similarity or machine learning}
  \begin{itemize}
  \item semantic similarity measures are mostly hand-crafted
    \begin{itemize}
    \item capture certain intuition about what constitutes
      ``similarity''
    \item different measures for different kinds of similarity
    \item usually interpretable (and explainable)
    \end{itemize}
    \pause
  \item machine learning methods are mostly data-driven
    \begin{itemize}
    \item the architecture of the model is still hand-crafted
    \item usually hard to interpret
    \end{itemize}
  \end{itemize}
\end{frame}

\begin{frame}
  \frametitle{Ontologies and graphs}
  \begin{itemize}
  \item semantic similarity measures {\em and machine learning models} on
    ontologies can be graph-based, feature-based, or model-based
    \begin{itemize}
    \item graph-based: ontology as a graph
    \item feature-based: extract (or obtain) features for
      classes/relations
    \item model-based: define similarity within (special) $\Sigma$-structures
    \end{itemize}
    \pause
  \item we may need to generate graphs from ontologies
    \begin{itemize}
    \item {\em is-a} relations are easy (this is just {\tt owl:subClassOf})
    \item how about {\em part-of}, {\em regulates}, {\em precedes},
      etc.?
    \item disjointness, universal vs. existential quantification,
      cardinality restrictions, intersection, union, negation?
    \end{itemize}
    \pause
  \item relational patterns are implicit in OWL axioms
    \begin{itemize}
    \item design patterns as ``relations'' between classes
    \end{itemize}
  \end{itemize}
\end{frame}

\begin{frame}
  \frametitle{Relations as patterns}
  \centerline{\includegraphics[height=.8\textheight]{plant-ontology-sample.png}}
\end{frame}

% \begin{frame}
%   \frametitle{Relations as patterns}
%   \begin{itemize}
%   \item OBO Relation Ontology (RO):
%     \begin{itemize}
%     \item \url{https://github.com/oborel/obo-relations}
%     \end{itemize}
%   \item Basic Formal Ontology (BFO):
%     \begin{itemize}
%     \item provides top-level classes
%       \begin{itemize}
%       \item Continuant, Process, Function, Material object, etc.
%       \end{itemize}
%     \item used for some OBO Foundry ontologies
%     \end{itemize}
%   \item RO and BFO provide a top-level system of classes and relations
%     shared across many biomedical ontologies
%   \item this system may define patterns used to generate graphs
%   \end{itemize}
% \end{frame}

\begin{frame}
  \frametitle{Relations as patterns}
  \begin{itemize}
  \item {\tt X SubClassOf: Y}: $X \xrightarrow{\text{is-a}} Y$
  \item {\tt X SubClassOf: part-of some Y}: $X \xrightarrow{\text{part-of}} Y$
  \item {\tt X SubClassOf: regulates some Y}: $X \xrightarrow{\text{regulates}} Y$
  \item {\tt X DisjointWith: Y}: $X \xleftrightarrow{\text{disjoint}} Y$
  \item {\tt X EquivalentTo: Y}: $X \xleftrightarrow{\equiv} Y$,
    $\{X,Y\}$
  \item ...
  \end{itemize}
  NB: in bio-ontologies, the OBO Relation Ontology defines these
  patterns
\end{frame}

\begin{frame}
  \frametitle{Asserted and inferred}
  \begin{itemize}
  \item relation patterns can be asserted or inferred
  \item {\tt X SubClassOf: part-of some Y}
  \item {\tt Y SubClassOf: part-of some Z}
  \item {\tt part-of o part-of SubPropertyOf: part-of}
  \item $\vdash$ {\tt X SubClassOf: part-of some Z}
  \item Therefore: $X \xrightarrow{\text{part-of}} Z$
  \item $\Rightarrow$ we should use deductive inference to generate
    these patterns
  \end{itemize}
\end{frame}

\begin{frame}
  \frametitle{Tree models}
  \begin{itemize}
  \item some languages have the {\em finite model} and {\em tree
      model} properties
    \begin{itemize}
    \item such as the Description Logic $\mathcal{ALC}$
    \item generated through a tableaux algorithm
    \end{itemize}
  \item nodes: individuals
    \begin{itemize}
    \item node labels: concept names, concept descriptions
    \end{itemize}
  \item edges: relations between individuals
  \item can be extended to more expressive languages (with blocking,
    cycles, etc.)
  \end{itemize}
\end{frame}

\begin{frame}
  \frametitle{Methods and tools}
  \begin{itemize}
  \item edges should be ``meaningful'': not merely syntax (why?)
    \begin{itemize}
    \item the RDF serialization of OWL is a graph and contains all
      information but is a bad idea for semantic similarity or machine
      learning (more later)
    \item conceptual graphs?
    \end{itemize}
  \item OBO Format represents ontologies as graphs:
    \begin{itemize}
    \item Protege/OWLAPI: OBO export
    \item OBO toolsets (e.g., ROBOT)
    \item
      \url{https://github.com/bio-ontology-research-group/Onto2Graph}
    \end{itemize}
    \pause
  \item but: a conversion of an ontologies into a graph will almost
    always lead to a loss of information
  \end{itemize}
\end{frame}

\section{Semantic similarity}

\begin{frame}
  Semantic similarity
  \begin{itemize}
  \item We want to use {\em  background knowledge} in ontologies to
    \begin{itemize}
    \item determine similarity between classes,
    \item instances,
    \item and entities with ontology annotations
    \end{itemize}
  \end{itemize}
\end{frame}


\begin{frame}
  \frametitle{How to measure similarity?}
  \begin{itemize}
  \item semantic similarity measures similarity between classes
  \item semantic similarity measures similarity between instances of classes
  \item semantic similarity measures similarity between entities
    {\em annotated} with classes
  \item $\Rightarrow$ reduce all of this to similarity between classes
  \end{itemize}
\end{frame}

\begin{frame}
  \frametitle{How to measure similarity?}
  What properties do we want in a similarity measure?
  \\
  A function $sim: D \times D$ is a similarity on $D$ if, for
  all $x, y \in D$, the function $sim$ is:  \begin{itemize}
    \pause
  \item non-negative: $sim(x,y) \geq 0$ for all $x, y$
    \pause
  \item symmetric: $sim(x,y) = sim(y,x)$
    \pause
  \item reflexive: $sim(x,x) = max_D$
    \pause
    \begin{itemize}
    \item weaker form: $sim(x,x) > sim(x,y)$ for all $x \not= y$
    \end{itemize}
    \pause
  \item $sim(x,x) > sim(x,y)$ for $x\not= y$
    \pause
  \item $sim$ is a {\em normalized} similarity measure if it has
    values in $[0,1]$
  \end{itemize}
\end{frame}

\usetikzlibrary{arrows,positioning,automata}
\begin{frame}
  \frametitle{How to measure similarity?}
  \begin{columns}
    \begin{column}{.6\textwidth}
      {\tiny
        \begin{tikzpicture}[>=stealth',shorten >=1pt,node distance=2cm,on grid,initial/.style    ={}]
          \node[state]          (A)                        {$Thing$};
          \node[state]          (B) [below left =of A]    {$Color$};
          \node[state]          (C) [below right =of A]    {$Shape$};
          \node[state]          (D) [below left =of B]    {$Red$};
          \node[state]          (H) [below right =of B]    {$Green$};
          \node[state]          (E) [below  =of D]    {$Orange$};
          \node[state]          (F) [below =of C]    {$Round$};
          \node[state]          (G) [below right =of C]    {$Square$};
          \tikzset{mystyle/.style={->,double=orange}} 
          \tikzset{every node/.style={fill=white}} 
          \path (B)     edge [mystyle]    node   {$isa$} (A)
          (C)     edge [mystyle]    node   {$isa$} (A) 
          (D)     edge [mystyle]    node   {$isa$} (B)
          (H)     edge [mystyle]    node   {$isa$} (B)
          (E)     edge [mystyle]    node   {$isa$} (D)
          (F)     edge [mystyle]    node   {$isa$} (C)
          (G)     edge [mystyle]    node   {$isa$} (C);
          \tikzset{mystyle/.style={<->,double=orange}}   
          \tikzset{mystyle/.style={<->,relative=true,in=0,out=60,double=orange}}
        \end{tikzpicture}
      }
    \end{column}
    \begin{column}{.4\textwidth}
      \begin{itemize}
        \pause
      \item distance on shortest path (Rada {\em et al.}, 1989)
        \pause
      \item $dist_{Rada}(u,v) = sp(u, isa, v)$
        \pause
      \item $sim_{Rada}(u,v) = \frac{1}{dist_{Rada}(u,v) + 1}$
      \end{itemize}
    \end{column}
  \end{columns}
\end{frame}

\begin{frame}
  \frametitle{How to measure similarity?}
  \begin{columns}
    \begin{column}{.6\textwidth}
      {\tiny
        \begin{tikzpicture}[>=stealth',shorten >=1pt,node distance=2cm,on grid,initial/.style    ={}]
          \node[state]          (A)                        {$Thing$};
          \node[state]          (B) [below left =of A]    {$Color$};
          \node[state]          (C) [below right =of A]    {$Shape$};
          \node[state, fill=green]          (D) [below left =of B]    {$Red$};
          \node[state, fill=green]          (H) [below right =of B]    {$Green$};
          \node[state]          (E) [below  =of D]    {$Orange$};
          \node[state]          (F) [below =of C]    {$Round$};
          \node[state]          (G) [below right =of C]    {$Square$};
          \tikzset{mystyle/.style={->,double=orange}} 
          \tikzset{highlight/.style={->,double=green}} 
          \tikzset{every node/.style={fill=white}}
          \path (B)     edge [mystyle]    node   {$isa$} (A)
          (C)     edge [mystyle]    node   {$isa$} (A) 
          (D)     edge [highlight]    node   {$isa$} (B)
          (H)     edge [highlight]    node   {$isa$} (B)
          (E)     edge [mystyle]    node   {$isa$} (D)
          (F)     edge [mystyle]    node   {$isa$} (C)
          (G)     edge [mystyle]    node   {$isa$} (C);
          \tikzset{mystyle/.style={<->,double=orange}}   
          \tikzset{mystyle/.style={<->,relative=true,in=0,out=60,double=orange}}
        \end{tikzpicture}
      }
    \end{column}
    \begin{column}{.4\textwidth}
      \begin{itemize}
      \item distance on shortest path
        \pause
       \item distance(green, red) = 2
       \item $sim_{Rada}(green, red) = \frac{1}{3}$
      \end{itemize}
    \end{column}
  \end{columns}
\end{frame}

\begin{frame}
  \frametitle{How to measure similarity?}
  \begin{columns}
    \begin{column}{.6\textwidth}
      {\tiny
        \begin{tikzpicture}[>=stealth',shorten >=1pt,node distance=2cm,on grid,initial/.style    ={}]
          \node[state]          (A)                        {$Thing$};
          \node[state]          (B) [below left =of A]    {$Color$};
          \node[state]          (C) [below right =of A]    {$Shape$};
          \node[state]          (D) [below left =of B]    {$Red$};
          \node[state]          (H) [below right =of B]    {$Green$};
          \node[state]          (E) [below  =of D]    {$Orange$};
          \node[state, fill=green]          (F) [below =of C]    {$Round$};
          \node[state, fill=green]          (G) [below right =of C]    {$Square$};
          \tikzset{mystyle/.style={->,double=orange}} 
          \tikzset{highlight/.style={->,double=green}} 
          \tikzset{every node/.style={fill=white}}
          \path (B)     edge [mystyle]    node   {$isa$} (A)
          (C)     edge [mystyle]    node   {$isa$} (A) 
          (D)     edge [mystyle]    node   {$isa$} (B)
          (H)     edge [mystyle]    node   {$isa$} (B)
          (E)     edge [mystyle]    node   {$isa$} (D)
          (F)     edge [highlight]    node   {$isa$} (C)
          (G)     edge [highlight]    node   {$isa$} (C);
          \tikzset{mystyle/.style={<->,double=orange}}   
          \tikzset{mystyle/.style={<->,relative=true,in=0,out=60,double=orange}}
        \end{tikzpicture}
      }
    \end{column}
    \begin{column}{.4\textwidth}
      \begin{itemize}
       \item distance on shortest path
       \item distance(square, round) = 2
       \item $sim_{Rada}(square, round) = \frac{1}{3}$
      \end{itemize}
    \end{column}
  \end{columns}
\end{frame}

\begin{frame}
  \frametitle{How to measure similarity?}
  \begin{columns}
    \begin{column}{.6\textwidth}
      {\tiny
        \begin{tikzpicture}[>=stealth',shorten >=1pt,node distance=2cm,on grid,initial/.style    ={}]
          \node[state]          (A)                        {$Thing$};
          \node[state, fill=green]          (B) [below left =of A]    {$Color$};
          \node[state]          (C) [below right =of A]    {$Shape$};
          \node[state]          (D) [below left =of B]    {$Red$};
          \node[state]          (H) [below right =of B]    {$Green$};
          \node[state, fill=green]          (E) [below  =of D]    {$Orange$};
          \node[state]          (F) [below =of C]    {$Round$};
          \node[state]          (G) [below right =of C]    {$Square$};
          \tikzset{mystyle/.style={->,double=orange}} 
          \tikzset{highlight/.style={->,double=green}} 
          \tikzset{every node/.style={fill=white}}
          \path (B)     edge [mystyle]    node   {$isa$} (A)
          (C)     edge [mystyle]    node   {$isa$} (A) 
          (D)     edge [highlight]    node   {$isa$} (B)
          (H)     edge [mystyle]    node   {$isa$} (B)
          (E)     edge [highlight]    node   {$isa$} (D)
          (F)     edge [mystyle]    node   {$isa$} (C)
          (G)     edge [mystyle]    node   {$isa$} (C);
          \tikzset{mystyle/.style={<->,double=orange}}   
          \tikzset{mystyle/.style={<->,relative=true,in=0,out=60,double=orange}}
        \end{tikzpicture}
      }
    \end{column}
    \begin{column}{.4\textwidth}
      \begin{itemize}
       \item distance on shortest path
       \item distance(orange, color) = 2
       \item $sim_{Rada}(orange, color) = \frac{1}{3}$
      \end{itemize}
    \end{column}
  \end{columns}
\end{frame}

\begin{frame}
  \frametitle{How to measure similarity?}
  \begin{itemize}
  \item shortest path is not always intuitive
    \pause
  \item we need a way to determine {\em specificity} of a class
    \begin{itemize}
    \item number of ancestors
    \item number of children
    \item information content
    \end{itemize}
    \pause
  \item {\em density} of a branch in the ontology
    \begin{itemize}
    \item number of siblings
    \item information content
    \end{itemize}
    \pause
  \item account for different edge types
    \begin{itemize}
    \item non-uniform edge weighting
    \end{itemize}
  \end{itemize}
\end{frame}

\begin{frame}
  \frametitle{How to measure similarity}
  \begin{itemize}
  \item term specificity measure $\sigma: C \mapsto \mathbb{R}$:
    \begin{itemize}
    \item $x \sqsubseteq y \rightarrow \sigma(x) \geq \sigma(y)$
    \end{itemize}
    \pause
  \item intrinsic:
    \begin{itemize}
    \item $\sigma(x) = f(depth(x))$
    \item $\sigma(x) = f(A(x))$ (for ancestors $A(x)$)
    \item $\sigma(x) = f(D(x))$ (for descendants $D(x)$)
    \item many more, e.g., Zhou et al.: $\sigma(x) = k \cdot \Big( 1-\frac{\log
        |D(x)|}{\log |C|} \Big) + (1-k) \frac{\log depth(x)}{\log
        depth(G_T)} $
    \end{itemize}
    \pause
  \item extrinsic:
    \begin{itemize}
    \item $\sigma(x)$ defined as a function of instances (or annotations) $I$
      \begin{itemize}
      \item note: the number of instances monotonically decreases with
        increasing depth in taxonomies
      \end{itemize}
    \item Resnik 1995: $eIC_{Resnik}(x) = -\log p(x)$ (with $p(x) =
      \frac{|I(x)|}{|I|}$)
      \begin{itemize}
      \item in biology, one of the most popular specificity measure when
        annotations are present
      \end{itemize}
    \end{itemize}
  \end{itemize}
\end{frame}

\begin{frame}
  \frametitle{How to measure similarity?}
  \begin{columns}
    \begin{column}{.6\textwidth}
      {\tiny
        \begin{tikzpicture}[>=stealth',shorten >=1pt,node distance=2cm,on grid,initial/.style    ={}]
          \node[state,label=below:$0.0$]          (A)                        {$Thing$};
          \node[state,label=below:$1.0$]          (B) [below left =of A]    {$Color$};
          \node[state,label=right:$1.0$]          (C) [below right =of A]    {$Shape$};
          \node[state,label=right:$2.0$]          (D) [below left =of B]    {$Red$};
          \node[state,label=below:$2.0$]          (H) [below right =of B]    {$Green$};
          \node[state,label=below:$3.0$]          (E) [below  =of D]    {$Orange$};
          \node[state,label=below:$2.0$]          (F) [below =of C]    {$Round$};
          \node[state,label=below:$2.0$]          (G) [below right =of C]    {$Square$};
          \tikzset{mystyle/.style={->,double=orange}} 
          \tikzset{highlight/.style={->,double=green}} 
          \tikzset{every node/.style={fill=white}}
          \path (B)     edge [mystyle]    node   {$isa$} (A)
          (C)     edge [mystyle]    node   {$isa$} (A) 
          (D)     edge [mystyle]    node   {$isa$} (B)
          (H)     edge [mystyle]    node   {$isa$} (B)
          (E)     edge [mystyle]    node   {$isa$} (D)
          (F)     edge [mystyle]    node   {$isa$} (C)
          (G)     edge [mystyle]    node   {$isa$} (C);
          \tikzset{mystyle/.style={<->,double=orange}}   
          \tikzset{mystyle/.style={<->,relative=true,in=0,out=60,double=orange}}
        \end{tikzpicture}
      }
    \end{column}
    \begin{column}{.4\textwidth}
      \begin{itemize}
      \item Resnik 1995: similarity between $x$ and $y$ is the
        information content of the {\em most informative common
          ancestor}
      \end{itemize}
    \end{column}
  \end{columns}
\end{frame}

\begin{frame}
  \frametitle{How to measure similarity?}
  \begin{columns}
    \begin{column}{.6\textwidth}
      {\tiny
        \begin{tikzpicture}[>=stealth',shorten >=1pt,node distance=2cm,on grid,initial/.style    ={}]
          \node[state,label=below:$0.0$]          (A)                        {$Thing$};
          \node[state,label=below:$1.0$]          (B) [below left =of A]    {$Color$};
          \node[state,label=right:$1.0$]          (C) [below right =of A]    {$Shape$};
          \node[state,fill=green,label=right:$2.0$]          (D) [below left =of B]    {$Red$};
          \node[state,fill=green,label=below:$2.0$]          (H) [below right =of B]    {$Green$};
          \node[state,label=below:$3.0$]          (E) [below  =of D]    {$Orange$};
          \node[state,label=below:$2.0$]          (F) [below =of C]    {$Round$};
          \node[state,label=below:$2.0$]          (G) [below right =of C]    {$Square$};
          \tikzset{mystyle/.style={->,double=orange}} 
          \tikzset{highlight/.style={->,double=green}} 
          \tikzset{every node/.style={fill=white}}
          \path (B)     edge [mystyle]    node   {$isa$} (A)
          (C)     edge [mystyle]    node   {$isa$} (A) 
          (D)     edge [mystyle]    node   {$isa$} (B)
          (H)     edge [mystyle]    node   {$isa$} (B)
          (E)     edge [mystyle]    node   {$isa$} (D)
          (F)     edge [mystyle]    node   {$isa$} (C)
          (G)     edge [mystyle]    node   {$isa$} (C);
          \tikzset{mystyle/.style={<->,double=orange}}   
          \tikzset{mystyle/.style={<->,relative=true,in=0,out=60,double=orange}}
        \end{tikzpicture}
      }
    \end{column}
    \begin{column}{.4\textwidth}
      \begin{itemize}
      \item Resnik 1995: similarity between $x$ and $y$ is the
        information content of the {\em most informative common
          ancestor}
      \end{itemize}
    \end{column}
  \end{columns}
\end{frame}

\begin{frame}
  \frametitle{How to measure similarity?}
  \begin{columns}
    \begin{column}{.6\textwidth}
      {\tiny
        \begin{tikzpicture}[>=stealth',shorten >=1pt,node distance=2cm,on grid,initial/.style    ={}]
          \node[state,label=below:$0.0$]          (A)                        {$Thing$};
          \node[state,fill=yellow,label=below:$1.0$]          (B) [below left =of A]    {$Color$};
          \node[state,label=right:$1.0$]          (C) [below right =of A]    {$Shape$};
          \node[state,fill=green,label=right:$2.0$]          (D) [below left =of B]    {$Red$};
          \node[state,fill=green,label=below:$2.0$]          (H) [below right =of B]    {$Green$};
          \node[state,label=below:$3.0$]          (E) [below  =of D]    {$Orange$};
          \node[state,label=below:$2.0$]          (F) [below =of C]    {$Round$};
          \node[state,label=below:$2.0$]          (G) [below right =of C]    {$Square$};
          \tikzset{mystyle/.style={->,double=orange}} 
          \tikzset{highlight/.style={->,double=green}} 
          \tikzset{every node/.style={fill=white}}
          \path (B)     edge [mystyle]    node   {$isa$} (A)
          (C)     edge [mystyle]    node   {$isa$} (A) 
          (D)     edge [mystyle]    node   {$isa$} (B)
          (H)     edge [mystyle]    node   {$isa$} (B)
          (E)     edge [mystyle]    node   {$isa$} (D)
          (F)     edge [mystyle]    node   {$isa$} (C)
          (G)     edge [mystyle]    node   {$isa$} (C);
          \tikzset{mystyle/.style={<->,double=orange}}   
          \tikzset{mystyle/.style={<->,relative=true,in=0,out=60,double=orange}}
        \end{tikzpicture}
      }
    \end{column}
    \begin{column}{.4\textwidth}
      \begin{itemize}
      \item Resnik 1995: similarity between $x$ and $y$ is the
        information content of the {\em most informative common
          ancestor}
      \end{itemize}
    \end{column}
  \end{columns}
\end{frame}

\begin{frame}
  \frametitle{How to measure similarity?}
  \begin{columns}
    \begin{column}{.6\textwidth}
      {\tiny
        \begin{tikzpicture}[>=stealth',shorten >=1pt,node distance=2cm,on grid,initial/.style    ={}]
          \node[state,label=below:$0.0$]          (A)                        {$Thing$};
          \node[state,fill=yellow,label=below:$1.0$]          (B) [below left =of A]    {$Color$};
          \node[state,label=right:$1.0$]          (C) [below right =of A]    {$Shape$};
          \node[state,fill=green,label=right:$2.0$]          (D) [below left =of B]    {$Red$};
          \node[state,fill=green,label=below:$2.0$]          (H) [below right =of B]    {$Green$};
          \node[state,label=below:$3.0$]          (E) [below  =of D]    {$Orange$};
          \node[state,label=below:$2.0$]          (F) [below =of C]    {$Round$};
          \node[state,label=below:$2.0$]          (G) [below right =of C]    {$Square$};
          \tikzset{mystyle/.style={->,double=orange}} 
          \tikzset{highlight/.style={->,double=green}} 
          \tikzset{every node/.style={fill=white}}
          \path (B)     edge [mystyle]    node   {$isa$} (A)
          (C)     edge [mystyle]    node   {$isa$} (A) 
          (D)     edge [mystyle]    node   {$isa$} (B)
          (H)     edge [mystyle]    node   {$isa$} (B)
          (E)     edge [mystyle]    node   {$isa$} (D)
          (F)     edge [mystyle]    node   {$isa$} (C)
          (G)     edge [mystyle]    node   {$isa$} (C);
          \tikzset{mystyle/.style={<->,double=orange}}   
          \tikzset{mystyle/.style={<->,relative=true,in=0,out=60,double=orange}}
        \end{tikzpicture}
      }
    \end{column}
    \begin{column}{.4\textwidth}
      \begin{itemize}
      \item Resnik 1995: similarity between $x$ and $y$ is the
        information content of the {\em most informative common
          ancestor}
        \item $sim_{Resnik}(Green, Red) = 1.0$
      \end{itemize}
    \end{column}
  \end{columns}
\end{frame}

\begin{frame}
  \frametitle{How to measure similarity?}
  \begin{columns}
    \begin{column}{.6\textwidth}
      {\tiny
        \begin{tikzpicture}[>=stealth',shorten >=1pt,node distance=2cm,on grid,initial/.style    ={}]
          \node[state,label=below:$0.0$]          (A)                        {$Thing$};
          \node[state,fill=yellow,label=below:$1.0$]          (B) [below left =of A]    {$Color$};
          \node[state,label=right:$1.0$]          (C) [below right =of A]    {$Shape$};
          \node[state,label=right:$2.0$]          (D) [below left =of B]    {$Red$};
          \node[state,fill=green,label=below:$2.0$]          (H) [below right =of B]    {$Green$};
          \node[state,fill=green,label=below:$3.0$]          (E) [below  =of D]    {$Orange$};
          \node[state,label=below:$2.0$]          (F) [below =of C]    {$Round$};
          \node[state,label=below:$2.0$]          (G) [below right =of C]    {$Square$};
          \tikzset{mystyle/.style={->,double=orange}} 
          \tikzset{highlight/.style={->,double=green}} 
          \tikzset{every node/.style={fill=white}}
          \path (B)     edge [mystyle]    node   {$isa$} (A)
          (C)     edge [mystyle]    node   {$isa$} (A) 
          (D)     edge [mystyle]    node   {$isa$} (B)
          (H)     edge [mystyle]    node   {$isa$} (B)
          (E)     edge [mystyle]    node   {$isa$} (D)
          (F)     edge [mystyle]    node   {$isa$} (C)
          (G)     edge [mystyle]    node   {$isa$} (C);
          \tikzset{mystyle/.style={<->,double=orange}}   
          \tikzset{mystyle/.style={<->,relative=true,in=0,out=60,double=orange}}
        \end{tikzpicture}
      }
    \end{column}
    \begin{column}{.4\textwidth}
      \begin{itemize}
      \item Resnik 1995: similarity between $x$ and $y$ is the
        information content of the {\em most informative common
          ancestor}
        \item $sim_{Resnik}(Green, Orange) = 1.0$
      \end{itemize}
    \end{column}
  \end{columns}
\end{frame}

\begin{frame}
  \frametitle{How to measure similarity?}
  \begin{columns}
    \begin{column}{.6\textwidth}
      {\tiny
        \begin{tikzpicture}[>=stealth',shorten >=1pt,node distance=2cm,on grid,initial/.style    ={}]
          \node[state,fill=yellow,label=below:$0.0$]          (A)                        {$Thing$};
          \node[state,label=below:$1.0$]          (B) [below left =of A]    {$Color$};
          \node[state,label=right:$1.0$]          (C) [below right =of A]    {$Shape$};
          \node[state,label=right:$2.0$]          (D) [below left =of B]    {$Red$};
          \node[state,fill=green,label=below:$2.0$]          (H) [below right =of B]    {$Green$};
          \node[state,label=below:$3.0$]          (E) [below  =of D]    {$Orange$};
          \node[state,label=below:$2.0$]          (F) [below =of C]    {$Round$};
          \node[state,fill=green,label=below:$2.0$]          (G) [below right =of C]    {$Square$};
          \tikzset{mystyle/.style={->,double=orange}} 
          \tikzset{highlight/.style={->,double=green}} 
          \tikzset{every node/.style={fill=white}}
          \path (B)     edge [mystyle]    node   {$isa$} (A)
          (C)     edge [mystyle]    node   {$isa$} (A) 
          (D)     edge [mystyle]    node   {$isa$} (B)
          (H)     edge [mystyle]    node   {$isa$} (B)
          (E)     edge [mystyle]    node   {$isa$} (D)
          (F)     edge [mystyle]    node   {$isa$} (C)
          (G)     edge [mystyle]    node   {$isa$} (C);
          \tikzset{mystyle/.style={<->,double=orange}}   
          \tikzset{mystyle/.style={<->,relative=true,in=0,out=60,double=orange}}
        \end{tikzpicture}
      }
    \end{column}
    \begin{column}{.4\textwidth}
      \begin{itemize}
      \item Resnik 1995: similarity between $x$ and $y$ is the
        information content of the {\em most informative common
          ancestor}
        \item $sim_{Resnik}(Square, Orange) = 0.0$
      \end{itemize}
    \end{column}
  \end{columns}
\end{frame}

\begin{frame}
  \frametitle{How to measure similarity?}
  \begin{itemize}
  \item (Red, Green) and (Orange, Green) have the same similarity
  \item need to incorporate the specificity of the compared classes
  \end{itemize}
\end{frame}

\begin{frame}
  \frametitle{How to measure similarity?}
  \begin{columns}
    \begin{column}{.6\textwidth}
      {\tiny
        \begin{tikzpicture}[>=stealth',shorten >=1pt,node distance=2cm,on grid,initial/.style    ={}]
          \node[state,label=below:$0.0$]          (A)                        {$Thing$};
          \node[state,fill=yellow,label=below:$1.0$]          (B) [below left =of A]    {$Color$};
          \node[state,label=right:$1.0$]          (C) [below right =of A]    {$Shape$};
          \node[state,fill=green,label=right:$2.0$]          (D) [below left =of B]    {$Red$};
          \node[state,fill=green,label=below:$2.0$]          (H) [below right =of B]    {$Green$};
          \node[state,label=below:$3.0$]          (E) [below  =of D]    {$Orange$};
          \node[state,label=below:$2.0$]          (F) [below =of C]    {$Round$};
          \node[state,label=below:$2.0$]          (G) [below right =of C]    {$Square$};
          \tikzset{mystyle/.style={->,double=orange}} 
          \tikzset{highlight/.style={->,double=green}} 
          \tikzset{every node/.style={fill=white}}
          \path (B)     edge [mystyle]    node   {$isa$} (A)
          (C)     edge [mystyle]    node   {$isa$} (A) 
          (D)     edge [mystyle]    node   {$isa$} (B)
          (H)     edge [mystyle]    node   {$isa$} (B)
          (E)     edge [mystyle]    node   {$isa$} (D)
          (F)     edge [mystyle]    node   {$isa$} (C)
          (G)     edge [mystyle]    node   {$isa$} (C);
          \tikzset{mystyle/.style={<->,double=orange}}   
          \tikzset{mystyle/.style={<->,relative=true,in=0,out=60,double=orange}}
        \end{tikzpicture}
      }
    \end{column}
    \begin{column}{.4\textwidth}
      \begin{itemize}
      \item Lin 1998: $sim_{Lin}(x,y) = \frac{2\cdot
          IC(MICA(x,y))}{IC(x) + IC(y)}$
        \pause
      \item $sim_{Lin}(Green, Red) = 0.5$
      \end{itemize}
    \end{column}
  \end{columns}
\end{frame}

\begin{frame}
  \frametitle{How to measure similarity?}
  \begin{columns}
    \begin{column}{.6\textwidth}
      {\tiny
        \begin{tikzpicture}[>=stealth',shorten >=1pt,node distance=2cm,on grid,initial/.style    ={}]
          \node[state,label=below:$0.0$]          (A)                        {$Thing$};
          \node[state,fill=yellow,label=below:$1.0$]          (B) [below left =of A]    {$Color$};
          \node[state,label=right:$1.0$]          (C) [below right =of A]    {$Shape$};
          \node[state,label=right:$2.0$]          (D) [below left =of B]    {$Red$};
          \node[state,fill=green,label=below:$2.0$]          (H) [below right =of B]    {$Green$};
          \node[state,fill=green,label=below:$3.0$]          (E) [below  =of D]    {$Orange$};
          \node[state,label=below:$2.0$]          (F) [below =of C]    {$Round$};
          \node[state,label=below:$2.0$]          (G) [below right =of C]    {$Square$};
          \tikzset{mystyle/.style={->,double=orange}} 
          \tikzset{highlight/.style={->,double=green}} 
          \tikzset{every node/.style={fill=white}}
          \path (B)     edge [mystyle]    node   {$isa$} (A)
          (C)     edge [mystyle]    node   {$isa$} (A) 
          (D)     edge [mystyle]    node   {$isa$} (B)
          (H)     edge [mystyle]    node   {$isa$} (B)
          (E)     edge [mystyle]    node   {$isa$} (D)
          (F)     edge [mystyle]    node   {$isa$} (C)
          (G)     edge [mystyle]    node   {$isa$} (C);
          \tikzset{mystyle/.style={<->,double=orange}}   
          \tikzset{mystyle/.style={<->,relative=true,in=0,out=60,double=orange}}
        \end{tikzpicture}
      }
    \end{column}
    \begin{column}{.4\textwidth}
      \begin{itemize}
      \item Lin 1998: $sim_{Lin}(x,y) = \frac{2\cdot
          IC(MICA(x,y))}{IC(x) + IC(y)}$
      \item $sim_{Lin}(Green, Orange) = 0.4$
      \end{itemize}
    \end{column}
  \end{columns}
\end{frame}

\begin{frame}
  \frametitle{How to measure similarity?}
  \begin{itemize}
  \item many(!) others:
    \begin{itemize}
    \item Jiang \& Conrath 1997
    \item Mazandu \& Mulder 2013
    \item Schlicker et al. 2009
    \item ...
  \end{itemize}
  \end{itemize}
\end{frame}

\begin{frame}
  \frametitle{How to measure similarity?}
  \begin{itemize}
  \item we only looked at comparing pairs of classes
  \item mostly, we want to compare {\em sets} of classes
    \begin{itemize}
    \item set of GO annotations
    \item set of signs and symptoms
    \item set of phenotypes
    \end{itemize}
  \item two approaches:
    \begin{itemize}
    \item compare each class individually, then merge
    \item directly set-based similarity measures
    \end{itemize}
  \end{itemize}
\end{frame}

\begin{frame}
  \frametitle{How to measure similarity?}
  \begin{columns}
    \begin{column}{.6\textwidth}
      {\tiny
        \begin{tikzpicture}[>=stealth',shorten >=1pt,node distance=2cm,on grid,initial/.style    ={}]
          \node[state,label=below:$0.0$]          (A)                        {$Thing$};
          \node[state,label=below:$1.0$]          (B) [below left =of A]    {$Color$};
          \node[state,label=right:$1.0$]          (C) [below right =of A]    {$Shape$};
          \node[state,fill=gray,label=right:$2.0$]          (D) [below left =of B]    {$Red$};
          \node[state,label=below:$2.0$]          (H) [below right =of B]    {$Green$};
          \node[state,fill=green,label=below:$3.0$]          (E) [below  =of D]    {$Orange$};
          \node[state,fill=gray,label=below:$2.0$]          (F) [below =of C]    {$Round$};
          \node[state,fill=green,label=below:$2.0$]          (G) [below right =of C]    {$Square$};
          \tikzset{mystyle/.style={->,double=orange}} 
          \tikzset{highlight/.style={->,double=green}} 
          \tikzset{every node/.style={fill=white}}
          \path (B)     edge [mystyle]    node   {$isa$} (A)
          (C)     edge [mystyle]    node   {$isa$} (A) 
          (D)     edge [mystyle]    node   {$isa$} (B)
          (H)     edge [mystyle]    node   {$isa$} (B)
          (E)     edge [mystyle]    node   {$isa$} (D)
          (F)     edge [mystyle]    node   {$isa$} (C)
          (G)     edge [mystyle]    node   {$isa$} (C);
          \tikzset{mystyle/.style={<->,double=orange}}   
          \tikzset{mystyle/.style={<->,relative=true,in=0,out=60,double=orange}}
        \end{tikzpicture}
      }
    \end{column}
    \begin{column}{.4\textwidth}
      \begin{itemize}
      \item similarity between a square-and-orange thing and a
        round-and-red thing
        \pause
      \item Pesquita et al., 2007:
        $simGIC(X,Y) = \frac{\sum_{c \in A(X) \cap A(Y)}
          IC(c)}{\sum_{c \in A(X) \cup A(Y)} IC(c)}$
      \end{itemize}
    \end{column}
  \end{columns}
\end{frame}

\begin{frame}
  \frametitle{How to measure similarity?}
  \begin{columns}
    \begin{column}{.6\textwidth}
      {\tiny
        \begin{tikzpicture}[>=stealth',shorten >=1pt,node distance=2cm,on grid,initial/.style    ={}]
          \node[state,fill=pink,label=below:$0.0$]          (A)                        {$Thing$};
          \node[state,fill=pink,label=below:$1.0$]          (B) [below left =of A]    {$Color$};
          \node[state,fill=pink,label=right:$1.0$]          (C) [below right =of A]    {$Shape$};
          \node[state,fill=gray,label=right:$2.0$]          (D) [below left =of B]    {$Red$};
          \node[state,label=below:$2.0$]          (H) [below right =of B]    {$Green$};
          \node[state,fill=green,label=below:$3.0$]          (E) [below  =of D]    {$Orange$};
          \node[state,fill=gray,label=below:$2.0$]          (F) [below =of C]    {$Round$};
          \node[state,fill=green,label=below:$2.0$]          (G) [below right =of C]    {$Square$};
          \tikzset{mystyle/.style={->,double=orange}} 
          \tikzset{highlight/.style={->,double=green}} 
          \tikzset{every node/.style={fill=white}}
          \path (B)     edge [mystyle]    node   {$isa$} (A)
          (C)     edge [mystyle]    node   {$isa$} (A) 
          (D)     edge [mystyle]    node   {$isa$} (B)
          (H)     edge [mystyle]    node   {$isa$} (B)
          (E)     edge [mystyle]    node   {$isa$} (D)
          (F)     edge [mystyle]    node   {$isa$} (C)
          (G)     edge [mystyle]    node   {$isa$} (C);
          \tikzset{mystyle/.style={<->,double=orange}}   
          \tikzset{mystyle/.style={<->,relative=true,in=0,out=60,double=orange}}
        \end{tikzpicture}
      }
    \end{column}
    \begin{column}{.4\textwidth}
      \begin{itemize}
      \item similarity between a square-and-orange thing and a
        round-and-red thing
      \item Pesquita et al., 2007:
        $simGIC(X,Y) = \frac{\sum_{c \in A(X) \cap A(Y)}
          IC(c)}{\sum_{c \in A(X) \cup A(Y)} IC(c)}$
      \item $simGIC(so,rr) = \frac{2}{11}$
      \end{itemize}
    \end{column}
  \end{columns}
\end{frame}

\begin{frame}
  \frametitle{How to measure similarity?}
  \begin{itemize}
  \item alternatively: use different merging strategies
  \item common: average, maximum, {\bf best-matching average}
    \begin{itemize}
    \item Average: $sim_A(X,Y) = \frac{\sum_{x\in X} \sum_{y \in Y} sim(x,y)}{|X| \times |Y|}$
    \item Max average: $sim_{MA}(X,Y) = \frac{1}{|X|} \sum_{x\in X} \max_{y \in Y} sim(x,y)$
    \item Best match average: $sim_{BMA}(X,Y) = \frac{sim_{MA}(X,Y) + sim_{MA}(Y,X)}{2}$
    \end{itemize}
  \end{itemize}
\end{frame}

\begin{frame}
  \frametitle{How to measure similarity?}
  \begin{itemize}
  \item Semantic Measures Library:
    \begin{itemize}
    \item comprehensive Java library
    \item \url{http://www.semantic-measures-library.org/}
    \end{itemize}
  \item R packages: GOSim, GOSemSim, HPOSim, LSAfun,
    ontologySimilarity,...
  \item Python: sematch, fastsemsim (GO only)
  \end{itemize}
\end{frame}

\begin{frame}
  \frametitle{Applications of semantic similarity}
  \begin{itemize}
  \item no obvious choice of similarity measure
  \item depends on application
    \begin{itemize}
    \item e.g., predicting PPIs in different organisms through
      similarity may benefit from a different similarity measure!
    \end{itemize}
  \item different similarity measures may react differently to biases
    in data
  \item needs some testing and experience
  \end{itemize}
\end{frame}

\begin{frame}
  \frametitle{Applications of semantic similarity}
  Recommendations:
  \begin{itemize}
  \item use Resnik's information content measure
  \item use Resnik's similarity
  \item use Best Match Average
  \item use the full ontology
  \item classify your ontology using an automated reasoner before
    applying semantic similarity
    \begin{itemize}
    \item although many ontologies come pre-classified
    \end{itemize}
  \item $\Rightarrow$ but there are many exceptions
    \begin{itemize}
    \item similar location $\Rightarrow$ use location subset of GO
    \item developmental phenotypes $\Rightarrow$ use developmental
      branch of phenotype ontology
    \end{itemize}
  \end{itemize}
\end{frame}




% \begin{frame}
%   \frametitle{An example: protein--protein interactions and GO functions}
%   \centerline{\includegraphics[width=.49\textwidth]{YeastPPINetwork.png}
%     \includegraphics[width=.49\textwidth]{go-figure1.jpg}}
% \end{frame}

\section{Machine learning and ontologies}

\begin{frame}
  \frametitle{Machine learning with ontologies: approaches}
  \begin{itemize}
  \item syntactic: treat axioms as ``sentences'' using language models
  \item graph-based: treat ontologies as graphs (like in semantic similarity)
  \item model-theoretic: encode model-theoretic semantics in optimization
  \end{itemize}
\end{frame}

\subsection{Syntactic approaches}

\begin{frame}
  \frametitle{Ontologies: axioms, not graphs!}
    \includegraphics[width=1\textwidth]{bcellapoptosis.png}
\end{frame}

\begin{frame}
  \frametitle{Ontologies: axioms, not graphs!}
  Gene Ontology:
  \begin{itemize}
  \item {\tt behavior DisjointWith: 'developmental process'}
  \item {\tt behavior SubclassOf: only-in-taxon some metazoa}
  \item {\tt 'cell proliferation' DisjointWith: in-taxon some fungi}
  \item {\tt 'cell growth' EquivalentTo: growth and ('results in
      growth of' some cell)}
  \item ...
  \end{itemize}
\end{frame}

% \begin{frame}
%   \frametitle{Ontologies: axioms, not graphs!}
%   \begin{itemize}
%   \item converting ontologies to graphs
%     \begin{itemize}
%     \item loses information
%     \end{itemize}
%   \item relations between ontologies
%   \end{itemize}
% \end{frame}

\begin{frame}
  \frametitle{Ontology embeddings}
  \begin{definition}
    Let $O = (\Sigma = (C, R, I); ax; \vdash)$ be an ontology with a set of
    classes $C$, a set of relations $R$, a set of instances $I$, a set
    of axioms $ax$ and an inference relation $\vdash$. An ontology
    embedding is a function $f_\eta : C \cup R \cup I \mapsto
    \Re^n$ (or $\Sigma(O) \mapsto \Re^n$. % (subject to certain constraints).
  \end{definition}
  \pause For example, we can use co-occurrence within $ax^\vdash$ to
  constrain the embedding function, where the constraints on
  co-occurrence are formulated using the Word2Vec skipgram model.
\end{frame}

\begin{frame}
  \frametitle{Onto2Vec}
  \centerline{\includegraphics[width=\textwidth]{onto2vecflow.png}}
\end{frame}

\begin{frame}
  \frametitle{Word2Vec}
  Maximize:
  \begin{equation}
    \frac{1}{N} \sum_{n=1}^{N} \sum_{-c\leq j \leq c, j\not=
      0} \log p(w_{n+j}|w_n)
  \end{equation}
  with
  \begin{equation}
    p(w_o | w_i) = \frac{\exp({v'_{w_o}}^T v_{w_i})}{\sum_{w=1}^{W}
      \exp({v'_{w}}^T v_{w_i})}
  \end{equation}
  (at least conceptually; different strategies are used to approximate Eqn. 2)
\end{frame}

\begin{frame}
  \frametitle{Word2Vec}
  \centerline{\includegraphics[width=\textwidth]{word2vec-example.png}}
\end{frame}

\begin{frame}
  \frametitle{Predicting PPIs: trainable similarity measures}
  \centerline{\includegraphics[width=.45\textwidth]{YSTUnsuper1.png}\includegraphics[width=.45\textwidth]{YstUnsup2.png}}

   {\tiny Smaili et al. Onto2Vec: joint vector-based representation of
     biological entities and their ontology-based annotations.}
\end{frame}

\begin{frame}
  \frametitle{Visualizing embeddings}
  \centerline{\includegraphics[width=.9\textwidth]{updtsne.jpg}}
\end{frame}

\begin{frame}
  \frametitle{Combination with text}
  \begin{itemize}
  \item ontologies contain more than axioms:
    \begin{itemize}
    \item labels, synonyms, definitions, authors, etc.
    \end{itemize}
  \item Description Logic axioms != natural language
  \item transfer learning: learn on one domain/task, apply to another
    \begin{itemize}
    \item e.g.: learn on literature, apply to ontologies
    \item words have ``meaning'' in literature, Description Logic
      symbols have ``meaning'' in ontology axioms
    \end{itemize}
  \item Ontologies Plus Annotations 2 Vec (OPA2Vec) combines both
  \end{itemize}
\end{frame}

\begin{frame}
  \frametitle{Ontologies Plus Annotations 2 Vec}
  \centerline{\includegraphics[width=1\textwidth]{opaworkflow16.png}}
\end{frame}

\begin{frame}
  \frametitle{Axioms contribute to prediction tasks: GO and GO-PLUS}
    % \processtable
    % \caption{AUC values of ROC curves for PPI prediction for
    %   GO-Plus and GO using Onto2Vec (cosine similiarity) and
    %   Onto2Vec-NN (neural network).\label{Tab:GOplus}}
    { \begin{tabularx}{\columnwidth}{XXXXXXX}\toprule & {}& {} &
                                                                 Human & Yeast & Arabidopsis\\\midrule
        $GO\_Onto2Vec$ & {} &{}& 0.7660 & 0.7701 & 0.7559 \\
        $GO\_Onto2Vec\_NN$ & {} & {}& 0.8779& 0.8711 & 0.8364 \\
        $GO\_plus\_Onto2Vec$ & {} & {}& 0.7880& 0.7943 & 0.7889 \\
        $GO\_plus\_Onto2Vec\_NN$ &{} & {}& \textbf{0.9021}&\textbf{0.8937} & \textbf{0.8834}\\
        \hline
      \end{tabularx}}{}
\end{frame}

\begin{frame}
  \frametitle{Evaluating individual axioms}
%  Testing how much each ontologies' axioms contribute to predictions:
%  \begin{resizebox}{\textwidth}{!}{
  Testing how much each ontologies' axioms contribute to predictions:
  \tiny
      \begin{tabularx}{\linewidth}{X|XX|XX}
        \toprule
        {} & \multicolumn{2}{c}{\textbf{Human}} & \multicolumn{2}{c}{\textbf{Arabidopsis}}\\
        \midrule
        {} & \textbf{Onto2Vec}&\textbf{Onto2Vec\_NN} &\textbf{Onto2Vec}&\textbf{Onto2Vec\_NN} \\
        \midrule
        GO (Baseline) &0.7660 &0.8779  & 0.7559 & 0.8364 \\
        ChEBI &0.7899\textcolor{blue}{(+0.0239)}& 0.8914\textcolor{blue}{(+0.0135)}  & 0.7703\textcolor{blue}{(+0.0144)}& 0.8518\textcolor{blue}{(+0.0154)} \\
        PO &0.7752\textcolor{blue}{(+0.0092)} & 0.8776\textcolor{red}{(-0.0003)} & 0.7671\textcolor{blue}{(+0.0112)} & 0.8469\textcolor{blue}{(+0.0105)}\\
        CL & 0.7743\textcolor{blue}{(+0.0083)} & 0.8810\textcolor{blue}{(+0.0031)} & 0.7612\textcolor{blue}{(+0.0053)}& 0.8371\textcolor{blue}{(+0.0007)}\\
        PATO & 0.7657\textcolor{red}{(-0.0003)} & 0.8768\textcolor{red}{(-0.0011)} & 0.7563\textcolor{blue}{(+0.0004)} & 0.8380\textcolor{blue}{(+0.0016)}\\
      \end{tabularx}
%    }
%  \centerline{\includegraphics[width=1\textwidth]{pato-eval1.png}}
\end{frame}

\begin{frame}
  \frametitle{Evaluating definitions}
  Testing how much each ontologies' annotation properties contribute to predictions:
  \tiny
      \begin{tabularx}{\linewidth}{X|XX|XX}
        \toprule
        {} & \multicolumn{2}{c}{\textbf{Human}} & \multicolumn{2}{c}{\textbf{Arabidopsis}}\\
        \midrule
        {} & \textbf{Onto2Vec}&\textbf{Onto2Vec\_NN} &\textbf{Onto2Vec}&\textbf{Onto2Vec\_NN} \\
        \midrule
        GO (Baseline)&0.8727 &0.9033  & 0.8613 & 0.8903 \\
        ChEBI & 0.8571\textcolor{red}{(-0.0156)} &0.8801\textcolor{red}{(-0.0232)} &0.8601\textcolor{red}{(-0.0012)}& 0.8880\textcolor{red}{(-0.0023)}\\
        PO & 0.8680\textcolor{red}{(-0.0047)}&0.8824\textcolor{red}{(-0.0209)} & 0.8632\textcolor{blue}{(+0.0019)} & 0.8908\textcolor{blue}{(+0.0005)}\\
        CL & 0.8811\textcolor{blue}{(+0.0084)}&0.9037\textcolor{blue}{(+0.0004)}  & 0.8614\textcolor{blue}{(+0.0001)} & 0.8899\textcolor{red}{(-0.0004)}\\
        PATO & 0.8562\textcolor{red}{(-0.0165)}& 0.8711\textcolor{red}{(-0.0322)} & 0.8544\textcolor{red}{(-0.0069)}& 0.8860\textcolor{red}{(-0.0043)} \\
      \end{tabularx}
%  \centerline{\includegraphics[width=1\textwidth]{pato-eval2.png}}
\end{frame}

\begin{frame}
  \frametitle{OPA2Vec}
  \begin{itemize}
  \item \url{https://github.com/bio-ontology-research-group/opa2vec}
  \item command line tool
    \begin{itemize}
    \item input: OWL ontology, set of entities with annotations/associations
    \item output: vectors for each class and entity
    \end{itemize}
  \item includes Elk and HermiT
  \item limitations: word-based
    \begin{itemize}
    \item completely ignores any semantics!
    \end{itemize}
  \end{itemize}
\end{frame}

\begin{frame}
  \frametitle{OPA2Vec Jupyter excercise}
  \begin{itemize}
  \item open the notebook {\tt OPA2Vec.ipynb}
  \item run the whole notebook
    \begin{itemize}
    \item this should be relatively fast and not take too much time on
      a modern laptop
    \end{itemize}
  \item play with the prediction methods (cosine similarity)
  \end{itemize}
\end{frame}

\end{document}
%%% Local Variables:
%%% mode: latex
%%% TeX-master: t
%%% End:
